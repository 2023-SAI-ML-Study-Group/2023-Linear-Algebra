%
% This is the LaTeX template file for lecture notes for CS267,
% Applications of Parallel Computing.  When preparing 
% LaTeX notes for this class, please use this template.
%
% To familiarize yourself with this template, the body contains
% some examples of its use.  Look them over.  Then you can
% run LaTeX on this file.  After you have LaTeXed this file then
% you can look over the result either by printing it out with
% dvips or using xdvi.
%

\documentclass{article}
%% \setlength{\oddsidemargin}{-0.25 in}
%% %% \setlength{\evensidemargin}{-0.25 in}
%% \setlength{\topmargin}{-0.6 in}
%% \setlength{\textwidth}{6.5 in}
%% \setlength{\textheight}{8.5 in}
\usepackage[margin=1in]{geometry}
\setlength{\headsep}{0.75 in}
\setlength{\parindent}{0 in}
\setlength{\parskip}{0.1 in}

%
% ADD PACKAGES here:
%

\usepackage{amsmath,amsfonts,graphicx,mathtools}
\usepackage{algorithm,algorithmic,color}

%
% The following commands set up the lecnum (lecture number)
% counter and make various numbering schemes work relative
% to the lecture number.
%
\newcounter{lecnum}
\renewcommand{\thepage}{\thelecnum-\arabic{page}}
\renewcommand{\thesection}{\thelecnum.\arabic{section}}
\renewcommand{\theequation}{\thelecnum.\arabic{equation}}
\renewcommand{\thefigure}{\thelecnum.\arabic{figure}}
\renewcommand{\thetable}{\thelecnum.\arabic{table}}

%
% The following macro is used to generate the header.
%
\newcommand{\lecture}[4]{
   \pagestyle{myheadings}
   \thispagestyle{plain}
   \newpage
   \setcounter{lecnum}{#1}
   \setcounter{page}{1}
   \noindent
   \begin{center}
   \framebox{
      \vbox{\vspace{2mm}
    \hbox to 6.28in { {\bf Linear Algebra
                        \hfill Spring 2023} }
       \vspace{4mm}
       \hbox to 6.28in { {\Large \hfill Lecture #1: #2  \hfill} }
       \vspace{2mm}
       \hbox to 6.28in { {\it Lecturer: #3 \hfill Scribes: #4} }
      \vspace{2mm}}
   }
   \end{center}
   \markboth{Lecture #1: #2}{Lecture #1: #2}
   {\bf Note}: {\it LaTeX template courtesy of UC Berkeley EECS dept.}

   {\bf Disclaimer}: {\it These notes are based on the freely available online lectures by Prof. Gilbert Strang.
}

   \vspace*{4mm}
}

%
% Convention for citations is authors' initials followed by the year.
% For example, to cite a paper by Leighton and Maggs you would type
% \cite{LM89}, and to cite a paper by Strassen you would type \cite{S69}.
% (To avoid bibliography problems, for now we redefine the \cite command.)
% Also commands that create a suitable format for the reference list.
\renewcommand{\cite}[1]{[#1]}
\def\beginrefs{\begin{list}%
        {[\arabic{equation}]}{\usecounter{equation}
         \setlength{\leftmargin}{2.0truecm}\setlength{\labelsep}{0.4truecm}%
         \setlength{\labelwidth}{1.6truecm}}}
\def\endrefs{\end{list}}
\def\bibentry#1{\item[\hbox{[#1]}]}

%Use this command for a figure; it puts a figure in wherever you want it.
%usage: \fig{NUMBER}{SPACE-IN-INCHES}{CAPTION}
\newcommand{\fig}[3]{
			\vspace{#2}
			\begin{center}
			Figure \thelecnum.#1:~#3
			\end{center}
	}

 
% Use these for theorems, lemmas, proofs, etc.
\newtheorem{theorem}{Theorem}[lecnum]
\newtheorem{lemma}[theorem]{Lemma}
\newtheorem{proposition}[theorem]{Proposition}
\newtheorem{claim}[theorem]{Claim}
\newtheorem{corollary}[theorem]{Corollary}
\newtheorem{definition}[theorem]{Definition}
\newenvironment{proof}{{\bf Proof:}}{\hfill\rule{2mm}{2mm}}

% **** IF YOU WANT TO DEFINE ADDITIONAL MACROS FOR YOURSELF, PUT THEM HERE:

\begin{document}

%FILL IN THE RIGHT INFO.
%\lecture{**LECTURE-NUMBER**}{**DATE**}{**LECTURER**}{**SCRIBE**}
\lecture{2}{An Overview of Key Ideas
}{Prof. Gilbert Strang}{Taegwan Cha}
%\footnotetext{These notes are partially based on those of Nigel Mansell.}

% **** YOUR NOTES GO HERE:

% Some general latex examples and examples making use of the
% macros follow.  
%**** IN GENERAL, BE BRIEF. LONG SCRIBE NOTES, NO MATTER HOW WELL WRITTEN,
%**** ARE NEVER READ BY ANYBODY.

\section{Vectors}

What do you do with vectors? Take combinations.
We can multiply vectors by scalars, add, and subtract. Given vectors u, v
and w we can form the linear combination x1u + x2v + x3w = b. 


\begin{center}
\[
  u_{} = 
  \begin{bmatrix}
    1\\
    -1\\
    0\\
  \end{bmatrix}
    v_{} = 
  \begin{bmatrix}
    0\\
    1\\
    -1\\
  \end{bmatrix}
   w_{} = 
  \begin{bmatrix}
    0\\
    0\\
    1\\
  \end{bmatrix}
\]


\end{center}

The collection of all multiples of u forms a line through the origin. The collection of all multiples of v forms another line. The collection of all combinations
of u and v forms a plane. Taking all combinations of some vectors creates a
subspace

\subsection{Matrices}
Create a matrix A with vectors \textbf{u} , \textbf{v} and \textbf{w} in its columns:
 
\[
  A_{} = 
  \begin{bmatrix}
    1 & 0 & 0\\
    -1& 1 & 0\\
    0 & -1 & 1\\
  \end{bmatrix}
\]

\[
  AX_{} = 
  \begin{bmatrix}
    1 & 0 & 0\\
    -1& 1 & 0\\
    0 & -1 & 1\\
  \end{bmatrix}
  \times
  \begin{bmatrix}
   x_1\\
    x_2\\
    x_3\\ 
  \end{bmatrix}
  =
   \begin{bmatrix*}[r]
    x_1\\
    -x_1 + x_2\\
    -x_2 + x_3\\ 
  \end{bmatrix*}
 
\]

equals the sum x1u + x2v + x3w = b. \\\\
When we say x1u + x2v + x3w = b we’re thinking about multiplying numbers by vectors; when we say Ax = b we’re thinking about multiplying a
matrix (whose columns are u, v and w) by the numbers. The calculations are
the same, but our perspective has changed

\pagebreak
For any input vector x, the output of the operation “multiplication by A” is
some vector b:

\[
  A_{} 
  \begin{bmatrix}
    1\\
    4\\
    9\\
  \end{bmatrix}
  =
  \begin{bmatrix}
    1\\
    3\\
    5\\
  \end{bmatrix}
\]


A deeper question is to start with a vector b and ask “for what vectors x does
Ax = b?” In our example, this means solving three equations in three unknowns. Solving: 


 
\[
  AX_{} = 
  \begin{bmatrix}
    1 & 0 & 0\\
    -1& 1 & 0\\
    0 & -1 & 1\\
  \end{bmatrix}
  \times
  \begin{bmatrix}
   x_1\\
    x_2\\
    x_3\\ 
  \end{bmatrix}
  =
   \begin{bmatrix*}[r]
    x_1\\
    x_2-x_1\\
    x_3-x_2\\ 
  \end{bmatrix*}
  =
   \begin{bmatrix*}[r]
    b_1\\
    b_2\\
    b_3\\ 
  \end{bmatrix*}
\]

is equivalent to solving: 
\begin{center}
\begin{align}
    x_1 = b_1\nonumber\\
    x_2-x_1 = b_2\nonumber\\
    x_3-x_2 = b_3\nonumber\\
    \nonumber
\end{align}
\end{center}
We see that x1 = b1 and so x2 must equal b1 + b2. In vector form, the solution
is: 

\[
  _{}  
  \begin{bmatrix}
    x_1\\
    x_2\\
    x_3\\
  \end{bmatrix}
  =
   \begin{bmatrix*}[r]
    b_1\\
    b_1+b_2\\
    b_1+b_2+b_3\\ 
  \end{bmatrix*}
\]

But this just says: 


\[
  X_{}=  
  \begin{bmatrix}
    1 & 0 & 0\\
    1 & 1 & 0\\
    1 & 1 & 1\\
  \end{bmatrix}
   \begin{bmatrix*}[r]
    b_1\\
    b_2\\
    b_3\\ 
  \end{bmatrix*}
\]

\\\\or x = A−1b. If the matrix A is invertible, we can multiply on both sides by
A−1 to find the unique solution x to Ax = b. We might say that A represents a
transform x → b that has an inverse transform b x.


\[
  C_{} = 
  \begin{bmatrix}
    1 & 0 & -1\\
    -1& 1 & 0\\
    0 & -1 & 1\\
  \end{bmatrix}
\]
Then:
\[
  CX_{} = 
  \begin{bmatrix}
    1 & 0 & -1\\
    -1& 1 & 0\\
    0 & -1 & 1\\
  \end{bmatrix}
  \times
  \begin{bmatrix}
   x_1\\
    x_2\\
    x_3\\ 
  \end{bmatrix}
  =
   \begin{bmatrix*}[r]
    x_1-x_3\\
    x_2-x_1\\
    x_3-x_2 \\ 
  \end{bmatrix*}
 \\
\]
and our system of three equations in three unknowns becomes circular. 

This is \textbf{unsolveable}. But
\pagebreak
\\
Where before Ax = 0 implied x = 0, there are non-zero vectors x for which
Cx = 0. For any vector x with x1 = x2 = x3, Cx = 0. This is a significant
difference; we can’t multiply both sides of Cx = 0 by an inverse to find a nonzero solution x.
The system of equations encoded in Cx = b is:
\begin{center}
\begin{align}
    x_1-x_3 = b_1\nonumber\\
    x_2-x_1 = b_2\nonumber\\
    x_3-x_2 = b_3\nonumber\\
    \nonumber
\end{align}
\end{center}

If we add these three equations together, we get: 

\begin{center}
\begin{align}
    0 = b_1+b_2+b_3\nonumber\\
    \nonumber
\end{align}
\end{center}
\\
This tells us that Cx = b has a solution x only when the components of b sum to 0. In a physical system, this might tell us that the system is stable as long as
the forces on it are balanced. 
\\\\



\subsection{Subspaces}

Geometrically, the columns of C lie in the same plane (they are dependent; the
columns of A are independent). There are many vectors in R3 which do not lie
in that plane. Those vectors cannot be written as a linear combination of the
columns of C and so correspond to values of b for which Cx = b has no solution x. The linear combinations of the columns of C form a two dimensional
subspace of R3.\\\\
This plane of combinations of u, v and w can be described as “all vectors
Cx”. But we know that the vectors b for which Cx = b satisfy the condition
b1 + b2 + b3 = 0. So the plane of all combinations of u and v consists of all
vectors whose components sum to 0.
If we take all combinations of:
\\


\begin{center}
\[
  u_{} = 
  \begin{bmatrix}
    1\\
    -1\\
    0\\
  \end{bmatrix}
    v_{} = 
  \begin{bmatrix}
    0\\
    1\\
    -1\\
  \end{bmatrix}
   w_{} = 
  \begin{bmatrix}
    0\\
    0\\
    1\\
  \end{bmatrix}
\]
\end{center}

we get the entire space R3; the equation Ax = b has a solution for every b in
R3. We say that u, v and w form a basis for R3.\\
A basis for Rn is a collection of n independent vectors in Rn. Equivalently,
a basis is a collection of n vectors whose combinations cover the whole space.
Or, a collection of vectors forms a basis whenever a matrix which has those
vectors as its columns is invertible.\\
A vector space is a collection of vectors that is closed under linear combinations. A subspace is a vector space inside another vector space; a plane through
the origin in R3 is an example of a subspace. A subspace could be equal to the
space it’s contained in; the smallest subspace contains only the zero vector.
The subspaces of R3 are:

\texttt{itemize}
\begin{itemize}
    \item the Origin(dot)
    \item a line through the origin,
    \item a plane through the origin,
    \item all of R3.
\end{itemize}


\subsection{Conclusion}
When you look at a matrix, try to see “what is it doing?”\\
Matrices can be rectangular; we can have seven equations in three unknowns. Rectangular matrices are not invertible, but the symmetric, square
matrix ATA that often appears when studying rectangular matrices may be
invertible. 

\end{document}





